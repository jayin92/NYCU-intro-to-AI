\documentclass{article}[12pt]
\usepackage{fontspec}   %加這個就可以設定字體
\usepackage{xeCJK}       %讓中英文字體分開設置
\usepackage{indentfirst}
\usepackage{listings}
\usepackage[newfloat]{minted}
\usepackage{float}
\usepackage{graphicx}
\usepackage{caption}
\usepackage{fancyhdr}
\usepackage{hyperref}
\usepackage{amsmath}
\usepackage{multirow}
\usepackage[dvipsnames]{xcolor}
\usepackage{graphicx}
\usepackage{tabularx}
\usepackage{booktabs}
\usepackage{caption}
\usepackage{subcaption}


\usepackage[breakable, listings, skins, minted]{tcolorbox}
\usepackage{etoolbox}
\setminted{fontsize=\footnotesize}
\renewtcblisting{minted}{%
    listing engine=minted,
    minted language=python,
    listing only,
    breakable,
    enhanced,
    minted options = {
        linenos, 
        breaklines=true, 
        breakbefore=., 
        % fontsize=\footnotesize, 
        numbersep=2mm
    },
    overlay={%
        \begin{tcbclipinterior}
            \fill[gray!25] (frame.south west) rectangle ([xshift=4mm]frame.north west);
        \end{tcbclipinterior}
    }   
}

\usepackage[
  top=2cm,
  bottom=2cm,
  left=2cm,
  right=2cm,
  headheight=17pt, % as per the warning by fancyhdr
  includehead,includefoot,
  heightrounded, % to avoid spurious underfull messages
]{geometry} 

\newenvironment{code}{\captionsetup{type=listing}}{}
\SetupFloatingEnvironment{listing}{name=Code}


\title{人工智慧概論 HW2 報告}
\author{110550088 李杰穎}
\date{\today}

\setCJKmainfont{Noto Serif TC}
\setmonofont[Mapping=tex-text]{Consolas}

\XeTeXlinebreaklocale "zh"             %這兩行一定要加,中文才能自動換行
\XeTeXlinebreakskip = 0pt plus 1pt     %這兩行一定要加,中文才能自動換行

\setlength{\parindent}{0em}
\setlength{\parskip}{2em}
\renewcommand{\baselinestretch}{1.5}
\begin{document}

\maketitle

\section{Preprocessing}
在本次作業中,我主要實作了以下幾種 preprocessing 的方法,條列如下:
\begin{enumerate}
    \item 將英文大寫轉成小寫: 使用 Python 內建的 \texttt{lower()} 函式,將字串文字全部
    轉成小寫
    \item 移除 stopwords: 利用 nltk 提供的 stopwords 列表,將 stopwords 從字串中移除。
    \item 移除 \texttt{<br />} HTML tag: 使用 Python 內建的 \texttt{replace()} 函式,
    將 \texttt{<br />} 用一個空白取代
    \item 移除標點符號: 利用 for 迴圈檢查每一個 char 是否為標點符號,如果非標點符號則將其加進一個 list,
    最後再使用 \texttt{``".join()} 來將 list 內元素轉為字串。檢查標點符號的部分則使用內建之
    \texttt{string.punctuation} 來檢查。
    \item Stemming (使用 nltk 內建之 SnowballStemmer): Stemming (詞幹提取) 是一種將詞彙去除後綴的方式。
    將單字進行 stemming 會讓模型不用處理額外的訊息,以下是一些經過 SnowballStemmer 處理後的單字。\\
    cared $\rightarrow$         care, 
    university $\rightarrow$     univers, 
    fairly $\rightarrow$         fair, 
    easily $\rightarrow$         easili,
    singing $\rightarrow$       sing, 
    sings $\rightarrow$         sing,  
    sung  $\rightarrow$          sung,  
    singer $\rightarrow$        singer,  
    sportingly $\rightarrow$     sport
\end{enumerate}

在進行 preprocessing 時,會依照上面排列的順序進行這五個步驟。

下方為一英文句子通過以上 preprocessing 後的句子。

\small{``It is a truth universally acknowledged that<br /> 
a single man in possession of a good fortune must be in want of a wife." $\rightarrow$
``truth univers acknowledg singl man possess good fortun must want wife"}

我也會在下文討論各 preprocessing 的方法對於最終的 F1-Score 的影響。

\section{Implement the bi-gram language model}

本部分介紹 bi-gram model 的實作,因為大部分實作細節都可以在繳交的程式碼中看到,故在此
我只大致說明實作內容。

\begin{enumerate}
    \item 計算各 unigram 及 bigram 的出現頻率: 由於在計算 $P(w_i|w_{i-1})$ 時,需要同時知道 bigram 及
    unigram 的出現頻率,故先 iterate 所有 document 並利用 dict 來統計出現的頻率。
    \item 利用上一個步驟的頻率求出 $P(w_i|w_{i-1})$: 左述之條件機率可由下式得到:
    \begin{equation}
        \label{eq: P}
        P(w_i|w_{i-1}) = \frac{\text{count}(w_{i-1}, w_i)}{\text{count}(w_{i-1})}
    \end{equation}
    \item 將所有算出的機率利用 Python 內建的 dict 資料結構,
    存為 \texttt{model[$w_{i-1}$][$w_i$]} 的形式,而 feature 則儲存各 bigram
    的頻率 (即為 \texttt{feature[($w_{i-1}$, $w_i$)]})
\end{enumerate}

可以發現若 Eq. \ref{eq: P} 中分子及分母若為 0,則會使機率的計算發生問題,
於是我在此使用 Add-1 (Laplace) Smoothing 來避免以下問題,Add-1 Smoothing 的具體式子如下:

\begin{equation}
    \label{eq: add-1}
    P(w_i|w_{i-1}) = \frac{\text{count}(w_{i-1}, w_i) + 1}{\text{count}(w_{i-1}) + |V|}
\end{equation}

其中 $|V|$ 為 unique vocabulary 的數量。在使用 Add-1 Smoothing 後,就不會出現先前
機率出現 0 或無限大的情形。

\section{Implement feature selection and conversion}

我主要使用使用了兩種 feature selection 的方法,第一種方法是按 bigram 的
出現頻率進行排序。第二種則是 Chi-Square Test 進行 feature selection。第一種方式較為簡單,在此不在贅述。
接下來會介紹 Chi-Square Test 的實做細節。

\subsection{Chi-Square Test Feature Selection}
$\chi^2$ 是一個可以計算出特定 feature 對於最終答案 dependent 程度的演算法,對於二元分類問題
其計算方式如下:

\begin{align*}
    e_{00}(\text{bigram}) &= \frac{sum_{neg}(sum_{pos}+sum_{neg}-(N_{pos}(\text{bigram})+N_{neg}(\text{bigram})))}{sum_{pos}+sum_{neg}} \\
    e_{01}(\text{bigram}) &= \frac{sum_{pos}(sum_{pos}+sum_{neg}-(N_{pos}(\text{bigram})+N_{neg}(\text{bigram})))}{sum_{pos}+sum_{neg}} \\
    e_{10}(\text{bigram}) &= \frac{sum_{neg}(N_{pos}(\text{bigram})+N_{neg}(\text{bigram}))}{sum_{pos}+sum_{neg}} \\
    e_{11}(\text{bigram}) &= \frac{sum_{pos}(N_{pos}(\text{bigram})+N_{neg}(\text{bigram}))}{sum_{pos}+sum_{neg}} \\
\end{align*}
\begin{equation}
    \begin{aligned}
        \chi^2(\text{bigram}) = &\frac{(sum_{neg} - N_{neg}(\text{bigram}) - e_{00}(\text{bigram}))^2}{e_{00}(\text{bigram})} \\
        &+ \frac{(sum_{pos} - N_{pos}(\text{bigram}) - e_{01}(\text{bigram}))^2}{e_{01}(\text{bigram})} \\
        &+ \frac{(N_{neg}(\text{bigram}) - e_{10}(\text{bigram}))^2}{e_{10}(\text{bigram})} \\
        &+ \frac{(N_{pos}(\text{bigram}) - e_{11}(\text{bigram}))^2}{e_{11}(\text{bigram})}
    \end{aligned}    
\end{equation}

其中 $sum_{pos}$ 為 positive 句子中的 bigram 總和,
$sum_{neg}$ 為 negative 句子中的 bigram 總和。
$N_{pos}(\text{bigram})$ 為 $\text{bigram}$ 在 positive 句子出現的次數,
$N_{neg}(\text{bigram})$ 為 $\text{bigram}$ 在 negative 句子出現的次數。

透過以上方式可以計算出特定 bigram 的 $\chi^2$ score。$\chi^2$ score 越高,代表此 bigram
對於結果越 dependent,也就代表此 bigram 越有價值。

計算出所有 bigram 的 $\chi^2$ score 後,我們就可以將各 bigram 透過 $\chi^2$ score 從小
排到大,並取出前 \texttt{feature\_num} 個 bigram 做為 GaussianNB 的輸入。

\section{Perplexity}
透過 Eq. \ref{eq: add-1},我們即可以計算一個 bigram 模型的 perplexity,perplexity 的計算方式
如下:

\begin{equation}
    \text{Perplexity} = 2^{-\text{entropy}}, \text{where entropy} = \frac{1}{N} \sum^{N}_{i=0} \log_{2}(P(w_i|w_{i-1}))
\end{equation}

當我們使用測試資料去測試模型時,算出的 perplexity 越低時,代表模型認為測試資料中的句子出現的機率越高。
也就是說當一個語言模型的 perplexity 越低時,模型的 performance 越好。

\section{Experiments}

\subsection{Preprocessing 相關實驗}
此部分會探討不同 preprocessing 的方法對於 perplexity, F1-score, precision 及 recall
的影響。



\section{Discussion}
\subsection{}



\end{document}