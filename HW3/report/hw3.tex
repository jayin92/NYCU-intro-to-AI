\documentclass{article}[12pt]
\usepackage{fontspec}   %加這個就可以設定字體
\usepackage{xeCJK}       %讓中英文字體分開設置
\usepackage{indentfirst}
\usepackage{listings}
\usepackage[newfloat]{minted}
\usepackage{float}
\usepackage{graphicx}
\usepackage{caption}
\usepackage{fancyhdr}
\usepackage{hyperref}
\usepackage{amsmath}
\usepackage{multirow}
\usepackage[dvipsnames]{xcolor}
\usepackage{graphicx}
\usepackage{tabularx}
\usepackage{booktabs}
\usepackage{caption}
\usepackage{subcaption}
\usepackage{pifont}
\usepackage{amssymb}

\usepackage[breakable, listings, skins, minted]{tcolorbox}
\usepackage{etoolbox}
\setminted{fontsize=\footnotesize}
\renewtcblisting{minted}{%
    listing engine=minted,
    minted language=python,
    listing only,
    breakable,
    enhanced,
    minted options = {
        linenos, 
        breaklines=true, 
        breakbefore=., 
        % fontsize=\footnotesize, 
        numbersep=2mm
    },
    overlay={%
        \begin{tcbclipinterior}
            \fill[gray!25] (frame.south west) rectangle ([xshift=4mm]frame.north west);
        \end{tcbclipinterior}
    }   
}

\usepackage[
  top=2cm,
  bottom=2cm,
  left=3.5cm,
  right=3.5cm,
  headheight=17pt, % as per the warning by fancyhdr
  includehead,includefoot,
  heightrounded, % to avoid spurious underfull messages
]{geometry} 

\newenvironment{code}{\captionsetup{type=listing}}{}
\SetupFloatingEnvironment{listing}{name=Code}


\title{人工智慧概論 HW3 報告}
\author{110550088 李杰穎}
\date{\today}

\setCJKmainfont{Noto Serif TC}
\setmonofont[Mapping=tex-text]{Consolas}

\XeTeXlinebreaklocale "zh"             %這兩行一定要加,中文才能自動換行
\XeTeXlinebreakskip = 0pt plus 1pt     %這兩行一定要加,中文才能自動換行

\setlength{\parindent}{0em}
\setlength{\parskip}{2em}
\renewcommand{\baselinestretch}{1.5}
\begin{document}

\maketitle

\section{Adversarial Search}
\subsection{Implementation of Minimax and Expectimax Algorithms}
\begin{code}
\captionof{listing}{Minimax Algorithm}
\begin{minted}
class MinimaxAgent(MultiAgentSearchAgent):
    """
    Your minimax agent (par1-1)
    """
    def getAction(self, gameState):
        """
        Returns the minimax action from the current gameState using self.depth
        and self.evaluationFunction.

        Here are some method calls that might be useful when implementing minimax.

        gameState.getLegalActions(agentIndex):
        Returns a list of legal actions for an agent
        agentIndex=0 means Pacman, ghosts are >= 1

        gameState.getNextState(agentIndex, action):
        Returns the child game state after an agent takes an action

        gameState.getNumAgents():
        Returns the total number of agents in the game

        gameState.isWin():
        Returns whether or not the game state is a winning state

        gameState.isLose():
        Returns whether or not the game state is a losing state
        """
        "*** YOUR CODE HERE ***"
        # Begin your code
        actions = gameState.getLegalActions(0)
        candidates = []
        for action in actions:
            candidates.append((action, self.minimax(gameState.getNextState(0, action), self.depth-1, 1, False)))
        action, _ = max(candidates, key=lambda item: item[1][1])
        # print(f"action: {action}")
        return action
        # End your code
    def minimax(self, gameState, depth, agentIdx, maximize):
        if gameState.isWin() or gameState.isLose() or (depth == 0 and agentIdx == 0):
            return (gameState, self.evaluationFunction(gameState))
        actions = gameState.getLegalActions(agentIdx)
        candidates = []
        if maximize:
            for action in actions:
                candidates.append(self.minimax(gameState.getNextState(agentIdx, action), depth-1, 1, False))
            stateScore = max(candidates, key=lambda item: item[1])
            
        elif agentIdx < gameState.getNumAgents()-1:
            for action in actions:
                candidates.append(self.minimax(gameState.getNextState(agentIdx, action), depth, agentIdx+1, False))
            stateScore = min(candidates, key=lambda item: item[1])
        else:
            for action in actions:
                candidates.append(self.minimax(gameState.getNextState(agentIdx, action), depth, 0, True))
            stateScore = min(candidates, key=lambda item: item[1])
        
        return stateScore
\end{minted}
\end{code}
\subsection{Minimax 與 Expectimax 的比較}
\section{Q-learning}
\section{DQN}


\end{document}