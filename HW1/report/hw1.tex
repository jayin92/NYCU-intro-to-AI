\documentclass{article}[12pt]
\usepackage{fontspec}   %加這個就可以設定字體
\usepackage{xeCJK}       %讓中英文字體分開設置
\usepackage{indentfirst}
\usepackage{listings}
\usepackage[newfloat]{minted}
\usepackage{float}
\usepackage{graphicx}
\usepackage{caption}
\usepackage{fancyhdr}
\usepackage{hyperref}
\usepackage{amsmath}
\usepackage{multirow}
\usepackage[dvipsnames]{xcolor}
\usepackage{graphicx}
\usepackage{tabularx}


\usepackage[breakable, listings, skins, minted]{tcolorbox}
\usepackage{etoolbox}

\renewtcblisting{minted}{%
    listing engine=minted,
    minted language=python,
    listing only,
    breakable,
    enhanced,
    minted options = {
        linenos, 
        breaklines=true, 
        breakbefore=., 
        % fontsize=\footnotesize, 
        numbersep=2mm
    },
    overlay={%
        \begin{tcbclipinterior}
            \fill[gray!25] (frame.south west) rectangle ([xshift=4mm]frame.north west);
        \end{tcbclipinterior}
    }   
}

\usepackage[
  top=2cm,
  bottom=2cm,
  left=2cm,
  right=2cm,
  headheight=17pt, % as per the warning by fancyhdr
  includehead,includefoot,
  heightrounded, % to avoid spurious underfull messages
]{geometry} 

\newenvironment{code}{\captionsetup{type=listing}}{}
\SetupFloatingEnvironment{listing}{name=Code}


\title{Introduction to Artificial Intelligence HW 1 Report}
\author{110550088 李杰穎}
\date{\today}

\setCJKmainfont{Noto Serif TC}
\setmonofont[Mapping=tex-text]{CascadiaMono.ttf}

\XeTeXlinebreaklocale "zh"             %這兩行一定要加,中文才能自動換行
\XeTeXlinebreakskip = 0pt plus 1pt     %這兩行一定要加,中文才能自動換行

\setlength{\parindent}{0em}
\setlength{\parskip}{2em}
\renewcommand{\baselinestretch}{1.5}
\begin{document}

\maketitle

\section{Code and Explanation}
\begin{code}
\captionof{listing}{Part 1 (\texttt{datasets.py})}
\begin{minted}
import os
import cv2
import numpy as np
def loadImages(dataPath):
    """
    Load all Images in the folder and transfer a list of tuples. 
    The first element is the numpy array of shape (m, n) representing the image.
    (remember to resize and convert the parking space images to 36 x 16 grayscale images.) 
    The second element is its classification (1 or 0)
        Parameters:
        dataPath: The folder path.
        Returns:
        dataset: The list of tuples.
    """
    # Begin your code (Part 1)
    dataset = []
    for item in os.listdir(os.path.join(dataPath, "car")):
        img = cv2.imread(os.path.join(dataPath, "car", item))
        img = cv2.resize(img, (36, 16))
        img = cv2.cvtColor(img, cv2.COLOR_BGR2GRAY)
        data = (img, 1)
        dataset.append(data)
    
    for item in os.listdir(os.path.join(dataPath, "non-car")):
        img = cv2.imread(os.path.join(dataPath, "non-car", item))
        img = cv2.resize(img, (36, 16))
        img = cv2.cvtColor(img, cv2.COLOR_BGR2GRAY)
        data = (img, 0)
        dataset.append(data)
    # End your code (Part 1)
    
    return dataset
    
\end{minted}
\end{code}
\section{Results}

\section{Problems}
\subsection{The Performance of My Model is really bad}



\end{document}